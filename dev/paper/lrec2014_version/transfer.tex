\subsection{Transfer}

Serbo-Croatian and Slovenian are very closely related, and their
morphologies are very similar. Most of the transfer rules thus concern
the notational differences of the tagsets, or are written to cover
different word orders in the languages.

Following are examples of transfer rules, which also illustrate some
contrastive characteristics of the languages:

\begin{itemize}
% Futur
\item the future tense:
  \sentenceexample{
Gledal bom
$\leftrightarrow$ 
Gledat ću\footnote{The Serbo-Croatian analyser covers the encliticised future tense forms (gledat ću / gledaću) as well.}

[watch{\sc.lp.m.sg}][be{\sc.clt.p1.sg}] 
$\leftrightarrow$ 
[watch{\sc.inf}][will{\sc.clt.p1.sg}]

(I will watch.)
}

Both languages form the future tense in an analytic manner. While Slovenian
uses a perfective form of the verb \emph{to be} combined with the l-participle (analogous
to Serbo-Croatian future II), Serbo-Croatian uses a cliticised form of the verb \emph{to
  want} combined with the infinitive. Unlike the infinitive, the
l-participle carries the information on the gender and number. Since
in this simplest form we have no way of inferring the gender of the
subject in the direction Serbo-Croatian $\rightarrow$ Slovenian the translation defaults
to masculine.

\item \emph{lahko} and \emph{moći}:
  \sentenceexample{
    Bolezni lahko povzročijo virusi $\leftrightarrow$ Bolesti mogu prouzročiti virusi

    [Diseases{\sc.acc}] [easily{\sc.adv}] [cause{\sc.p3.sg}] [viruses{\sc.nom}] $\rightarrow$ [Diseases{\sc.acc}] [can{\sc.p3.sg}] [cause{\sc.inf}] [viruses{\sc.nom}]

    (Viruses can cause diseases.)
  }

Unlike its Serbo-Croatian cognate \emph{lako} the adverb \emph{lahko} in
Slovenian, when combined with a verb has an additional meaning of \emph{can be
  done}, expressed in Serbo-Croatian with the modal verb \emph{moći}. Rules that
cover these type of phrases normalise the target verb to infinitive,
and transfer grammatical markers for number and person to the verb \emph{moći}.

\item \emph{lahko} and conditional:
  \sentenceexample{
Lahko bi napravili $\leftrightarrow$ Mogli bi napraviti

[easily{\sc.adv}] [would{\sc.clt.cnd}] [do{\sc.lp.pl}] $\rightarrow$ [Can{\sc.lp.pl}] [would{\sc.clt.cnd.p3.sg}] [do{\sc.inf}]

(We/they could do)
}

Another morphological difference is found in the conditional
mood. The conditional marker in Serbo-Croatian is the aorist form of the verb
\emph{to be}, and carries the information on person and
number\footnote{\emph{bih}, \emph{bismo}, \emph{biste}, or \emph{bi}}. Slovenian, and the majority of colloquial Serbo-Croatian varieties, use
a frozen clitic form of the same verb.\footnote{\emph{bi} regardless of person and
number} Thus in cases like this example, when it is impossible to
exactly infer the person and number the system defaults to the
colloquial form.

\item \emph{lahko} and conditional more complicated:
  \sentenceexample{
Mi bi lahko napravili $\leftrightarrow$ Mi bismo mogli napraviti

[We{\sc.p1.pl}] [would{\sc.clt.cnd}] [easily{\sc.adv}] [do{\sc.lp.pl}]
$\rightarrow$ 
[We{\sc.p1.pl}] [would{\sc.clt.cnd.p3.pl}] [can{\sc.lp.pl}] [do{\sc.inf}]

(We could do)
}

The information on person and number is available on the pronoun
\emph{mi}, and can be copied in translation to the conditional verb.

\item \emph{treba} adverb to verb
  \sentenceexample{
    je treba narediti $\rightarrow$ treba učiniti

[is] [needed{\sc.adv}] [to be done{\sc.inf}]
$\rightarrow$ 
[needs{\sc.vb.p3.sg}] [to be done{\sc.inf}]

    (It needs to be done)
  }

Phrases with Slovenian adverb \emph{treba} translate to Serbo-Croatian with the
verb \emph{trebati}. In its simplest form the phrase just translates
as 3rd person singular.

For the opposite direction \emph{trebati} translates as the analogous
verb \emph{potrebovati}, so that no loss of morphological information occurs.

\sentenceexample{
trebaju našu solidarnost $\rightarrow$ potrebujejo našu solidarnost

(They need our solidarity)
  }

More complicated examples with different tenses and verb phrases involve word reodering:

\sentenceexample{
narediti je bilo treba $\leftrightarrow$ trebalo je napraviti

[do{\sc.inf}] [is{\sc.clt.p3.sg}] [was{\sc.lp.nt}] [need{\sc.adv}] 
$\rightarrow$
[needed{\sc.lp.nt}] [is{\sc.clt.p3.sg}] [do{\sc.inf}]

(It needed to be done.)
}

\end{itemize}
