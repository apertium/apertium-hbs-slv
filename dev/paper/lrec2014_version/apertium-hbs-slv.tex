\documentclass[10pt, a4paper]{article}
\usepackage{lrec2006}
\usepackage{graphicx}

\usepackage{url}

\usepackage[utf8]{inputenc}
\usepackage{multirow} 
\usepackage[small,bf]{caption} 

%% \usepackage{multibib}
%%\usepackage[comma]{natbib}
%% \usepackage{chapterbib}

\usepackage{lingmacros}
\usepackage{threeparttable}

\usepackage{fancyvrb} 
\DefineVerbatimEnvironment{outputExample}{Verbatim}{fontsize=\small}
\DefineVerbatimEnvironment{goldExample}{BVerbatim}{fontsize=\tiny}

\bibdata{apertium-hbs-slv}

\newcommand{\sentenceexample}[1]{{\small\enumsentence{#1}}}

\title{Shallow-transfer rule-based machine translation for the Western group of South Slavic languages}

%\name{Hrvoje Peradin, Filip Petkovsky, Francis M. Tyers}
\name{X, Y, Z}

\address{A, B, C}
%\address{University of Zagreb, University of Zagreb, Institut for språkvitskap \\
%               Faculty of Science, Faculty of Electrical Engineering, Det humanistiske fakultet \\
%               Dept. of Mathematics, and Computer Science, N-9037 Universitetet i Tromsø\\
%               hperadin@gmail.com, filip.petkovski@fer.hr, ftyers@prompsit.com\\}


\abstract{
The South Slavic languages, spoken mostly in the Balkans, make up one
of the three Slavic branches. The South Slavic branch is in turn
comprised of two subgroups, the Eastern subgroup containing Macedonian
and Bulgarian, and the western subgroup containing Serbo-Croatian
and Slovenian. This paper describes the development of a
bidirectional machine translation system for the western branch of
South-Slavic languages — Serbo-Croatian and Slovenian. Both
languages have a free word order, are highly inflected, and share a
great degree of mutual inteligibility. They are also under-resourced
as regards free/open-source resources. We give details on the resources and
development methods used, as well as an evaluation, and
general directions for future work.
 \\ \newline %\Keywords{keyword A, keyword B, keyword C}
}

\begin{document}

\maketitleabstract

\section{Introduction}
\todo{On Slovene}

\todo{On Serbo-Croatian}

\begin{figure}

\todo{Venn diagram of Serbo-Croatian, Serbian, Croatian, Bosnian, Montenegrin, 
Neo-Štokavian, Čakavian, Kajkavian, Torlakian
Ekavian, Ijekavian, Ikavian}

\end{figure}

\todo{On their relationship}


\section{Design}
\subsection{The Apertium platform}
\nocite{forcada2011apertium}
The Apertium\footnote{\url{http://wiki.apertium.org/}} platform is a
modular machine translation system. The typical core layout consists
of a letter transducer morphological lexicon.\footnote{A list of
ordered pairs of word surface forms and their lemmatised
analyses.} The transducer produces cohorts\footnote{A cohort 
consists of a surface form and one or more readings containing the lemma of the 
word and the morphological analysis.} which are then subjected to a
morphological disambiguation process.
%
Disambiguated readings are then looked up in the bilingual dictionary,
which gives the possible translations for each reading. These
are then passed through a lexical-selection module \cite{tyers12a}, 
which applies rules that select the most appropriate translation
for a given source-language context.
After lexical selection, the readings, which are now pairs of source
and target language lexical forms are passed through a 
syntactic transfer module that performs word reordering, deletions,
insertions, and basic syntactic chunking.
%
The final module is another letter transducer which generates
surface forms in the target language from the bilingual transfer
output cohorts.

\subsection{Constraint Grammar}
This language pair uses a Constraint Grammar (CG)
module\footnote{Implemented in the CG3 formalism, using the
  \texttt{vislcg3} compiler, available under GNU GPL. For a detailed
  reference see: \url{http://beta.visl.sdu.dk/cg3.html}} for
disambiguation \cite{karlsson1995constraint}. The CG formalism consists of hand-written rules that
are applied to a stream of tokens. Depending on the morphosyntactic
context of a given token the rules select or exclude readings of a
given surface form, or assign additional tags.


% Development; also contains disambiguation, lexical transfer, and structural transfer
\section{Development}

\subsection{Resources}
This language pair was developed with the
aid of on-line resources containing word definitions and flective
paradigms, such as \emph{Hrvatski jezični
  portal}\footnote{\url{http://hjp.srce.hr}} for the Serbo-Croatian side. For
the Slovene side we used a similar online resource \emph{Slovar
  slovenskega knjižnega
  jezika},\footnote{\url{http://bos.zrc-sazu.si/sskj.html}}, and the
\emph{Amebis Besana} flective
lexicon.\footnote{\url{http://besana.amebis.si/pregibanje/}}

The bilingual dictionary for the language pair was developed from scratch,
using the \emph{EUDict}\footnote{\url{http://eudict.com/}} online
dictionary and other online resources.
%\emph{Google Translate}\footnote{\url{http://translate.google.com/}}.

\subsection{Morphological analysis and generation}
The basis for this language pair are the morphological
lexicons for Serbo-Croatian (from
the language pair Serbo-Croatian--Macedonian, {\small{\tt apertium-hbs-mak}}) and Slovene (from the
language pair Slovene--Spanish, {\small{\tt apertium-sl-es}}). Both
lexicons are written in the XML formalism of
\emph{lttoolbox}\footnote{\url{http://wiki.apertium.org/wiki/Lttoolbox}}
(\citealp{rojas2005construccion}), and were developed as parts of
their respective language pairs, during the Google Summer of Code
2011.\footnote{\url{http://code.google.com/soc/}}. Since the lexicons
had been developed using different frequency lists, and slightly
different tagsets, they have been further trimmed and updated to
synchronise their coverage.

%wiktionaries and Wikipedia, as well as an SETimes corpus\footnote{\url{http://opus.lingfil.uu.se/SETIMES.php}} (\citealp{tyers2010south}) and a%corpus composed from the Serbian, Bosnian, Croatian and Serbo-Croatian Wikipedias.

%% Other resources for morphological analysis of Serbian and Croatian exist
%% (\citealp{vitas2004intex}, \citealp{vitas2003processing}, \citealp{agic2008improving}, \citealp{snajder08automatic}), 
%% to our knowledge there are none freely available for either Serbian, Bosnian or
%% Croatian. 

\subsection{Disambiguation}
Though for both languages there exists a number of tools for
morphological tagging and disambiguation \cite{vitas2004intex,agic2008improving,snajder08automatic}, there
are none freely available. Likewise, the adequately tagged corpora are
mostly non-free \cite{erjavec2004multext,tadic2002building}.  Since both Serbo-Croatian and Slovenian are
highly inflected languages, the automatically trained statistical
tagger canonically used in Apertium language pairs would not give
satisfactory results. For this reason we chose to use solely
Constraint Grammar (CG) for disambiguation. The CG module does not
provide complete disambiguation, so in the case of any remaining
ambiguity the system picks the first output analysis.
Due to the similarities between the languages, we were able to
reuse some of the rules developed earlier for Serbo-Croatian.

% [Hrvoje: commented out for SALTMIL]
%%Following are a few disambiguation rules examples:
%%
%%\begin{itemize}
%%
%%\item Simple adverb vs. adjective rule:
%%
%%\sentenceexample{
%%Ja često jedem ribu. $\leftrightarrow$ Jaz pogosto jedem ribo.
%%
%%(I eat fish often.)
%%}
%%
%%For this phrase the morphological analyser gives:
%%
%%{\small
%%\begin{Verbatim}
%%"<Ja>"
%%    "free" prn pers p1 mfn sg nom 
%%"<često>"
%%    "često" adv
%%;   "čest" adj pst nt sg nom ind
%%;   "čest" adj pst nt sg nom def
%%;   "čest" adj pst nt sg voc ind
%%;   "čest" adj pst nt sg voc def
%;   "čest" adj pst nt sg acc ind
%% ;   "čest" adj pst nt sg acc def
%% "<jedem>"
%%     "jesti" vb imperf tv pres p1 sg
%% ;   "jesti" vb imperf ref pres p1 sg
%% ;   "jesti" vb imperf iv pres p1 sg
%% "<ribu>"
%%      "riba" n f sg acc 
%% \end{Verbatim}
%% }

%% The rule used to disambiguate the adverb \emph{često} in this phrase

%% {\small\begin{verbatim}
%%     SELECT Adv IF 
%%         (0 Adv OR A) 
%%         (1C V)
%% \end{verbatim}
%% }

%% selects the adverb reading in an adverb/adjective ambiguity if the
%% word following is unambiguously a verb.

%% \item Preposition based case disambiguation:

%% \sentenceexample{
%% Za našo ljubo staro mater. $\leftrightarrow$ Za našu dragu staru majku.

%% (For our dear old mother.)
%% }

%% Noun phrases in both languages typically generate a great number of
%% ambiguities.

%% {\small
%% \begin{Verbatim}
%%     "<Za>"
%%          "za" pr acc 
%%     ;    "za" pr gen
%%     ;    "za" pr ins
%%     "<našo>"
%%          "naš" prn pos p1 f sg acc 
%%     ;    "naš" prn pos p1 f sg ins
%%     "<ljubo>"
%%          "ljub" adj f sg acc ind 
%%     ;    "ljubo" adv sint
%%     ;    "ljub" adj nt sg nom ind
%%     ;    "ljub" adj nt sg acc ind
%%     ;    "ljub" adj f sg ins ind
%%     "<staro>"
%%          "star" adj f sg acc ind 
%%     ;    "staro" adv sint
%%     ;    "star" adj f sg ins ind
%%     ;    "star" adj nt sg nom ind
%%     ;    "star" adj nt sg acc ind
%%     "<mater>"
%%          "mati" n f sg acc 
%%     ;    "mati" n f du gen
%%     ;    "mati" n f pl gen
%% \end{Verbatim} 
%% }

%% First the rule

%% {\small
%% \begin{Verbatim}
%%     REMOVE Pr + $$Case IF 
%%         (1 Nominal - $$Case)
%% \end{Verbatim}
%% }

%% removes a case reading from a preposition if it is not followed by an
%% adjective, noun or a pronoun in the same case, and then the rule

%% {\small
%% \begin{Verbatim}
%%     REMOVE Nominal + $$Case IF
%%         (NOT -1 Prep + $$Case) 
%%         (NOT -1 Modifier + $$Case)
%% \end{Verbatim}
%% }

%% in several passes removes the incorrect cases from the nouns and
%% modifiers in the phrase.

%% \item Noun heuristics:

%% \sentenceexample{
%% ...izvršavanje dužnosti predstavnice...

%% (...doing the duty of a representative...)
%% }

%% Frequently in both languages if there is a sequence of nouns, the
%% second noun is in genitive.

%% {\small
%% \begin{Verbatim}
%%     "<izvršavanje>"
%%         "izvršavanje" n nt sg nom 
%%     ;   "izvršavanje" n nt sg voc
%%     ;   "izvršavanje" n nt sg acc
%%     "<dužnosti>"
%%         "dužnost" n f sg gen
%%     ;   "dužnost" n f sg voc
%%     ;   "dužnost" n f pl voc
%%     ;   "dužnost" n f sg loc
%%     ;   "dužnost" n f sg dat
%%     ;   "dužnost" n f pl acc
%%     ;   "dužnost" n f sg ins
%%     ;   "dužnost" n f pl nom
%%     ;   "dužnost" n f pl gen
%%     "<predstavnice>"
%%         "predstavnica" n f sg gen
%%     ;   "predstavnica" n f sg voc
%%     ;   "predstavnica" n f pl voc
%%     ;   "predstavnica" n f pl nom
%%     ;   "predstavnica" n f pl acc
%% \end{Verbatim}
%% }

%% This simple heuristic selects a genitive reading of a noun after a noun:

%% {\small
%% \begin{Verbatim}
%%     SELECT: N + Gen IF (-1 N)
%% \end{Verbatim}
%% }
%% \end{itemize}



\subsection{Lexical transfer}
The lexical transfer was done with an \emph{lttoolbox} letter
transducer composed of bilingual dictionary entries. Additional
paradigms were added to the transducer to compensate for the tagset
notational differences.

\subsection{Lexical selection}

Since there was no adequate and free Slovenian -- Serbo-Croatian parallel corpus, 
we chose to do the lexical selection relying only on hand-written rules in 
Apertium's lexical selection module \citep{tyers12a}.
For cases not covered by our hand-written rules, the system would choose the 
default translation from the bilingual dictionary.
We provide examples of such lexical selection rules.

Phonetics based lexical selection: many words from the Croatian and Serbian dialects differ in a single phoneme.
An example are the words \emph{točno} in Croatian and \emph{tačno} in Serbian (engl. \emph{accurate}).
Such differences were solved through the lexical selection module using rules like:

{\small
\begin{Verbatim}
<rule>
  <match lemma="točno" tags="adv.*">
    <select lemma="točno" tags="adv.*"/>
  </match>
</rule>
\end{Verbatim}
}
for Croatian, and
{\small
\begin{Verbatim}
<rule>
  <match lemma="točno" tags="adv.*">
    <select lemma="tačno" tags="adv.*"/>
  </match>
</rule>
\end{Verbatim}
}
for Serbian and Bosnian.

Similarly, the Croatian language has the form \emph{burza} (meaning stock exchange in English), while Serbian and Bosnian have \emph{berza}. 
For those forms the following rules were written:

{\small
\begin{Verbatim}
<rule>
  <match lemma="borza" tags="n.*">
    <select lemma="burza" tags="n.*"/>
  </match>
</rule>
\end{Verbatim}
}
for Croatian, and 
{\small
\begin{Verbatim}
<rule>
  <match lemma="borza" tags="n.*">
    <select lemma="berza" tags="n.*"/>
  </match>
</rule>

\end{Verbatim}
}

for Serbian and Bosnian.

Another example of a phonetical difference are words which have h in Croatian and Bosnian, but v in Serbian.
Such words include \emph{kuha} and \emph{duhan} in Croatian and Bosnian, but \emph{kuva} and \emph{duvan} in Serbian.
Similar rules were written for the forms for \emph{porcelain} (procelan in Serbian and porculan in Croatian), 
\emph{salt} (so and sol) etc.

While the Serbian dialect accepts the Ekavian and Ikavian reflexes, 
the Croatian dialect uses only the Ijekavian reflex.
Since the selection for the different reflexes of the yat vowel is done in the generation process,
no rules were needed in the lexical selection module.

Internationalisms have been introduced to Croatian and Bosnian mainly through the Italian and German language
whereas they have entered Serbian through French and Russian. 
As a result, the three dialects have developed different phonetic patterns for internatonal words.

Examples of rules for covering such varieties include:
{\small
\begin{Verbatim}
<rule>
  <match lemma="Betlehem" tags="np.*">
    <select lemma="Betlehem" tags="np.*"/>
  </match>
</rule>
\end{Verbatim}
}
for Croatian and Bosnian, and
{\small
\begin{Verbatim}
<rule>
  <match lemma="Betlehem" tags="np.*">
    <select lemma="Vitlejem" tags="np.*"/>
  </match>
</rule>
\end{Verbatim}
}
for Serbian.

Finally, the Croatian months used for the Gregorian calendar have Slavic-derived names and differ from the original Latin names.
For example, the Croatian language has the word \emph{siječanj} for \emph{January}, and 
the Serbian language has the word \emph{Januar}.
These differences were also covered by the lexical selection module.

Besides the yat reflex, several other cases were not covered in the lexical selection module. These include the pronoun `what' which has the form \emph{što} in Croatian and the forms \emph{što} and \emph{šta} in Serbian and Bosnian, depending on whether the context is interrogative or relative. This ambiguity was left out of the lexical selection module and was dealth with during generation.





%% Redefiniramo enum sentence da bude manji, jer primjeri ispadaju preveliki

\subsection{Transfer}

Serbo-Croatian and Slovenian are very closely related, and their
morphologies are very similar. Most of the transfer rules written
are thus either technical, or written to cover different word order of
the languages.

Following are examples of transfer rules, which also illustrate some
contrastive characteristics of the languages:

\begin{itemize}
% Futur
\item the future tense:
  \sentenceexample{
Gledal bom
$\leftrightarrow$ 
Gledat ću\footnote{The Serbo-Croatian analyser covers the encliticised future tense forms (gledat ću / gledaću) as well.}

[watch{\sc.lp.m.sg}][be{\sc.clt.p1.sg}] 
$\leftrightarrow$ 
[watch{\sc.inf}][will{\sc.clt.p1.sg}]

(I will watch.)
}

Both languages form the future tense in an analytic manner. While Slovene
uses a perfective form of the verb \emph{to be} combined with the l-participle (analogous
to Serbo-Croatian future II), Serbo-Croatian uses a cliticised form of the verb \emph{to
  want} combined with the infinitive. Unlike the infinitive, the
l-participle carries the information on the gender and number. Since
in this simplest form we have no way of inferring the gender of the
subject in the direction Serbo-Croatian $\rightarrow$ Slovene the translation defaults
to masculine.

\item \emph{lahko} and \emph{moći}:
  \sentenceexample{
    Bolezni lahko povzročijo virusi $\leftrightarrow$ Bolesti mogu prouzročiti virusi

    [Diseases{\sc.acc}] [easily{\sc.adv}] [cause{\sc.p3.sg}] [viruses{\sc.nom}] $\rightarrow$ [Diseases{\sc.acc}] [can{\sc.p3.sg}] [cause{\sc.inf}] [viruses{\sc.nom}]

    (Viruses can cause diseases.)
  }

Unlike its Serbo-Croatian cognate \emph{lako} the adverb \emph{lahko} in
Slovene, when combined with a verb has an additional meaning of \emph{can be
  done}, expressed in Serbo-Croatian with the modal verb \emph{moći}. Rules that
cover these type of phrases normalise the target verb to infinitive,
and transfer grammatical markers for number and person to the verb \emph{moći}.

\item \emph{lahko} and conditional:
  \sentenceexample{
Lahko bi napravili $\leftrightarrow$ Mogli bi napraviti

[easily{\sc.adv}] [would{\sc.clt.cnd}] [do{\sc.lp.pl}] $\rightarrow$ [Can{\sc.lp.pl}] [would{\sc.clt.cnd.p3.sg}] [do{\sc.inf}]

(We/they could do)
}

Another case of morphological inequity is the conditional
mood. The conditional marker in Serbo-Croatian is the aorist form of the verb
\emph{to be}, and carries the information on person and
number\footnote{\emph{bih}, \emph{bismo}, \emph{biste}, or \emph{bi}}. Slovene, and the majority of colloquial Serbo-Croatian varieties, use
a frozen clitic form of the same verb\footnote{\emph{bi}, regardless of person and
number}. Thus in cases like this example, when it is impossible to
exactly infer the person and number the system defaults to the
colloquial form.

\item \emph{lahko} and conditional more complicated:
  \sentenceexample{
Mi bi lahko napravili $\leftrightarrow$ Mi bismo mogli napraviti

[We{\sc.p1.pl}] [would{\sc.clt.cnd}] [easily{\sc.adv}] [do{\sc.lp.pl}]
$\rightarrow$ 
[We{\sc.p1.pl}] [would{\sc.clt.cnd.p3.pl}] [can{\sc.lp.pl}] [do{\sc.inf}]

(We could do)
}

In this example the information on person and number is available
on the pronoun \emph{mi}, and can be copied in translation to the
conditional verb.

\item \emph{treba} adverb to verb
  \sentenceexample{
    je treba narediti $\rightarrow$ treba učiniti

[is] [needed{\sc.adv}] [to be done{\sc.inf}]
$\rightarrow$ 
[needs{\sc.vb.p3.sg}] [to be done{\sc.inf}]

    (It needs to be done)
  }

Phrases with Slovene adverb \emph{treba} translate to Serbo-Croatian with the
verb \emph{trebati}. In its simplest form the phrase just translates
as 3rd person singular.

For the opposite direction \emph{trebati} translates as the analogous
verb \emph{potrebovati}, so that no loss of morphological information occurs.

\sentenceexample{
trebaju našu solidarnost $\rightarrow$ potrebujejo našu solidarnost

(They need our solidarity)
  }

More complicated examples with different tenses and verb phrases involve word reodering:

\sentenceexample{
narediti je bilo treba $\leftrightarrow$ trebalo je napraviti

[do{\sc.inf}] [is{\sc.clt.p3.sg}] [was{\sc.lp.nt}] [need{\sc.adv}] 
$\rightarrow$
[needed{\sc.lp.nt}] [is{\sc.clt.p3.sg}] [do{\sc.inf}]

(It needed to be done.)
}

\end{itemize}



\subsection{Disambiguation}
Though for both languages there exists a number of tools for
morphological tagging and disambiguation \cite{vitas2004intex,agic2008improving,snajder08automatic}, there
are none freely available. Likewise, the adequately tagged corpora are
mostly non-free \cite{erjavec2004multext,tadic2002building}.  Since both Serbo-Croatian and Slovenian are
highly inflected languages, the automatically trained statistical
tagger canonically used in Apertium language pairs would not give
satisfactory results. For this reason we chose to use solely
Constraint Grammar (CG) for disambiguation. The CG module does not
provide complete disambiguation, so in the case of any remaining
ambiguity the system picks the first output analysis.
Due to the similarities between the languages, we were able to
reuse some of the rules developed earlier for Serbo-Croatian.

% [Hrvoje: commented out for SALTMIL]
%%Following are a few disambiguation rules examples:
%%
%%\begin{itemize}
%%
%%\item Simple adverb vs. adjective rule:
%%
%%\sentenceexample{
%%Ja često jedem ribu. $\leftrightarrow$ Jaz pogosto jedem ribo.
%%
%%(I eat fish often.)
%%}
%%
%%For this phrase the morphological analyser gives:
%%
%%{\small
%%\begin{Verbatim}
%%"<Ja>"
%%    "free" prn pers p1 mfn sg nom 
%%"<često>"
%%    "često" adv
%%;   "čest" adj pst nt sg nom ind
%%;   "čest" adj pst nt sg nom def
%%;   "čest" adj pst nt sg voc ind
%%;   "čest" adj pst nt sg voc def
%;   "čest" adj pst nt sg acc ind
%% ;   "čest" adj pst nt sg acc def
%% "<jedem>"
%%     "jesti" vb imperf tv pres p1 sg
%% ;   "jesti" vb imperf ref pres p1 sg
%% ;   "jesti" vb imperf iv pres p1 sg
%% "<ribu>"
%%      "riba" n f sg acc 
%% \end{Verbatim}
%% }

%% The rule used to disambiguate the adverb \emph{često} in this phrase

%% {\small\begin{verbatim}
%%     SELECT Adv IF 
%%         (0 Adv OR A) 
%%         (1C V)
%% \end{verbatim}
%% }

%% selects the adverb reading in an adverb/adjective ambiguity if the
%% word following is unambiguously a verb.

%% \item Preposition based case disambiguation:

%% \sentenceexample{
%% Za našo ljubo staro mater. $\leftrightarrow$ Za našu dragu staru majku.

%% (For our dear old mother.)
%% }

%% Noun phrases in both languages typically generate a great number of
%% ambiguities.

%% {\small
%% \begin{Verbatim}
%%     "<Za>"
%%          "za" pr acc 
%%     ;    "za" pr gen
%%     ;    "za" pr ins
%%     "<našo>"
%%          "naš" prn pos p1 f sg acc 
%%     ;    "naš" prn pos p1 f sg ins
%%     "<ljubo>"
%%          "ljub" adj f sg acc ind 
%%     ;    "ljubo" adv sint
%%     ;    "ljub" adj nt sg nom ind
%%     ;    "ljub" adj nt sg acc ind
%%     ;    "ljub" adj f sg ins ind
%%     "<staro>"
%%          "star" adj f sg acc ind 
%%     ;    "staro" adv sint
%%     ;    "star" adj f sg ins ind
%%     ;    "star" adj nt sg nom ind
%%     ;    "star" adj nt sg acc ind
%%     "<mater>"
%%          "mati" n f sg acc 
%%     ;    "mati" n f du gen
%%     ;    "mati" n f pl gen
%% \end{Verbatim} 
%% }

%% First the rule

%% {\small
%% \begin{Verbatim}
%%     REMOVE Pr + $$Case IF 
%%         (1 Nominal - $$Case)
%% \end{Verbatim}
%% }

%% removes a case reading from a preposition if it is not followed by an
%% adjective, noun or a pronoun in the same case, and then the rule

%% {\small
%% \begin{Verbatim}
%%     REMOVE Nominal + $$Case IF
%%         (NOT -1 Prep + $$Case) 
%%         (NOT -1 Modifier + $$Case)
%% \end{Verbatim}
%% }

%% in several passes removes the incorrect cases from the nouns and
%% modifiers in the phrase.

%% \item Noun heuristics:

%% \sentenceexample{
%% ...izvršavanje dužnosti predstavnice...

%% (...doing the duty of a representative...)
%% }

%% Frequently in both languages if there is a sequence of nouns, the
%% second noun is in genitive.

%% {\small
%% \begin{Verbatim}
%%     "<izvršavanje>"
%%         "izvršavanje" n nt sg nom 
%%     ;   "izvršavanje" n nt sg voc
%%     ;   "izvršavanje" n nt sg acc
%%     "<dužnosti>"
%%         "dužnost" n f sg gen
%%     ;   "dužnost" n f sg voc
%%     ;   "dužnost" n f pl voc
%%     ;   "dužnost" n f sg loc
%%     ;   "dužnost" n f sg dat
%%     ;   "dužnost" n f pl acc
%%     ;   "dužnost" n f sg ins
%%     ;   "dužnost" n f pl nom
%%     ;   "dužnost" n f pl gen
%%     "<predstavnice>"
%%         "predstavnica" n f sg gen
%%     ;   "predstavnica" n f sg voc
%%     ;   "predstavnica" n f pl voc
%%     ;   "predstavnica" n f pl nom
%%     ;   "predstavnica" n f pl acc
%% \end{Verbatim}
%% }

%% This simple heuristic selects a genitive reading of a noun after a noun:

%% {\small
%% \begin{Verbatim}
%%     SELECT: N + Gen IF (-1 N)
%% \end{Verbatim}
%% }
%% \end{itemize}



\subsection{Lexical transfer}
The lexical transfer was done with an \emph{lttoolbox} letter
transducer composed of bilingual dictionary entries. Additional
paradigms were added to the transducer to compensate for the tagset
notational differences.

\subsection{Lexical selection}

Since there was no adequate and free Slovenian -- Serbo-Croatian parallel corpus, 
we chose to do the lexical selection relying only on hand-written rules in 
Apertium's lexical selection module \citep{tyers12a}.
For cases not covered by our hand-written rules, the system would choose the 
default translation from the bilingual dictionary.
We provide examples of such lexical selection rules.

Phonetics based lexical selection: many words from the Croatian and Serbian dialects differ in a single phoneme.
An example are the words \emph{točno} in Croatian and \emph{tačno} in Serbian (engl. \emph{accurate}).
Such differences were solved through the lexical selection module using rules like:

{\small
\begin{Verbatim}
<rule>
  <match lemma="točno" tags="adv.*">
    <select lemma="točno" tags="adv.*"/>
  </match>
</rule>
\end{Verbatim}
}
for Croatian, and
{\small
\begin{Verbatim}
<rule>
  <match lemma="točno" tags="adv.*">
    <select lemma="tačno" tags="adv.*"/>
  </match>
</rule>
\end{Verbatim}
}
for Serbian and Bosnian.

Similarly, the Croatian language has the form \emph{burza} (meaning stock exchange in English), while Serbian and Bosnian have \emph{berza}. 
For those forms the following rules were written:

{\small
\begin{Verbatim}
<rule>
  <match lemma="borza" tags="n.*">
    <select lemma="burza" tags="n.*"/>
  </match>
</rule>
\end{Verbatim}
}
for Croatian, and 
{\small
\begin{Verbatim}
<rule>
  <match lemma="borza" tags="n.*">
    <select lemma="berza" tags="n.*"/>
  </match>
</rule>

\end{Verbatim}
}

for Serbian and Bosnian.

Another example of a phonetical difference are words which have h in Croatian and Bosnian, but v in Serbian.
Such words include \emph{kuha} and \emph{duhan} in Croatian and Bosnian, but \emph{kuva} and \emph{duvan} in Serbian.
Similar rules were written for the forms for \emph{porcelain} (procelan in Serbian and porculan in Croatian), 
\emph{salt} (so and sol) etc.

While the Serbian dialect accepts the Ekavian and Ikavian reflexes, 
the Croatian dialect uses only the Ijekavian reflex.
Since the selection for the different reflexes of the yat vowel is done in the generation process,
no rules were needed in the lexical selection module.

Internationalisms have been introduced to Croatian and Bosnian mainly through the Italian and German language
whereas they have entered Serbian through French and Russian. 
As a result, the three dialects have developed different phonetic patterns for internatonal words.

Examples of rules for covering such varieties include:
{\small
\begin{Verbatim}
<rule>
  <match lemma="Betlehem" tags="np.*">
    <select lemma="Betlehem" tags="np.*"/>
  </match>
</rule>
\end{Verbatim}
}
for Croatian and Bosnian, and
{\small
\begin{Verbatim}
<rule>
  <match lemma="Betlehem" tags="np.*">
    <select lemma="Vitlejem" tags="np.*"/>
  </match>
</rule>
\end{Verbatim}
}
for Serbian.

Finally, the Croatian months used for the Gregorian calendar have Slavic-derived names and differ from the original Latin names.
For example, the Croatian language has the word \emph{siječanj} for \emph{January}, and 
the Serbian language has the word \emph{Januar}.
These differences were also covered by the lexical selection module.

Besides the yat reflex, several other cases were not covered in the lexical selection module. These include the pronoun `what' which has the form \emph{što} in Croatian and the forms \emph{što} and \emph{šta} in Serbian and Bosnian, depending on whether the context is interrogative or relative. This ambiguity was left out of the lexical selection module and was dealth with during generation.




%% Redefiniramo enum sentence da bude manji, jer primjeri ispadaju preveliki

\subsection{Transfer}

Serbo-Croatian and Slovenian are very closely related, and their
morphologies are very similar. Most of the transfer rules written
are thus either technical, or written to cover different word order of
the languages.

Following are examples of transfer rules, which also illustrate some
contrastive characteristics of the languages:

\begin{itemize}
% Futur
\item the future tense:
  \sentenceexample{
Gledal bom
$\leftrightarrow$ 
Gledat ću\footnote{The Serbo-Croatian analyser covers the encliticised future tense forms (gledat ću / gledaću) as well.}

[watch{\sc.lp.m.sg}][be{\sc.clt.p1.sg}] 
$\leftrightarrow$ 
[watch{\sc.inf}][will{\sc.clt.p1.sg}]

(I will watch.)
}

Both languages form the future tense in an analytic manner. While Slovene
uses a perfective form of the verb \emph{to be} combined with the l-participle (analogous
to Serbo-Croatian future II), Serbo-Croatian uses a cliticised form of the verb \emph{to
  want} combined with the infinitive. Unlike the infinitive, the
l-participle carries the information on the gender and number. Since
in this simplest form we have no way of inferring the gender of the
subject in the direction Serbo-Croatian $\rightarrow$ Slovene the translation defaults
to masculine.

\item \emph{lahko} and \emph{moći}:
  \sentenceexample{
    Bolezni lahko povzročijo virusi $\leftrightarrow$ Bolesti mogu prouzročiti virusi

    [Diseases{\sc.acc}] [easily{\sc.adv}] [cause{\sc.p3.sg}] [viruses{\sc.nom}] $\rightarrow$ [Diseases{\sc.acc}] [can{\sc.p3.sg}] [cause{\sc.inf}] [viruses{\sc.nom}]

    (Viruses can cause diseases.)
  }

Unlike its Serbo-Croatian cognate \emph{lako} the adverb \emph{lahko} in
Slovene, when combined with a verb has an additional meaning of \emph{can be
  done}, expressed in Serbo-Croatian with the modal verb \emph{moći}. Rules that
cover these type of phrases normalise the target verb to infinitive,
and transfer grammatical markers for number and person to the verb \emph{moći}.

\item \emph{lahko} and conditional:
  \sentenceexample{
Lahko bi napravili $\leftrightarrow$ Mogli bi napraviti

[easily{\sc.adv}] [would{\sc.clt.cnd}] [do{\sc.lp.pl}] $\rightarrow$ [Can{\sc.lp.pl}] [would{\sc.clt.cnd.p3.sg}] [do{\sc.inf}]

(We/they could do)
}

Another case of morphological inequity is the conditional
mood. The conditional marker in Serbo-Croatian is the aorist form of the verb
\emph{to be}, and carries the information on person and
number\footnote{\emph{bih}, \emph{bismo}, \emph{biste}, or \emph{bi}}. Slovene, and the majority of colloquial Serbo-Croatian varieties, use
a frozen clitic form of the same verb\footnote{\emph{bi}, regardless of person and
number}. Thus in cases like this example, when it is impossible to
exactly infer the person and number the system defaults to the
colloquial form.

\item \emph{lahko} and conditional more complicated:
  \sentenceexample{
Mi bi lahko napravili $\leftrightarrow$ Mi bismo mogli napraviti

[We{\sc.p1.pl}] [would{\sc.clt.cnd}] [easily{\sc.adv}] [do{\sc.lp.pl}]
$\rightarrow$ 
[We{\sc.p1.pl}] [would{\sc.clt.cnd.p3.pl}] [can{\sc.lp.pl}] [do{\sc.inf}]

(We could do)
}

In this example the information on person and number is available
on the pronoun \emph{mi}, and can be copied in translation to the
conditional verb.

\item \emph{treba} adverb to verb
  \sentenceexample{
    je treba narediti $\rightarrow$ treba učiniti

[is] [needed{\sc.adv}] [to be done{\sc.inf}]
$\rightarrow$ 
[needs{\sc.vb.p3.sg}] [to be done{\sc.inf}]

    (It needs to be done)
  }

Phrases with Slovene adverb \emph{treba} translate to Serbo-Croatian with the
verb \emph{trebati}. In its simplest form the phrase just translates
as 3rd person singular.

For the opposite direction \emph{trebati} translates as the analogous
verb \emph{potrebovati}, so that no loss of morphological information occurs.

\sentenceexample{
trebaju našu solidarnost $\rightarrow$ potrebujejo našu solidarnost

(They need our solidarity)
  }

More complicated examples with different tenses and verb phrases involve word reodering:

\sentenceexample{
narediti je bilo treba $\leftrightarrow$ trebalo je napraviti

[do{\sc.inf}] [is{\sc.clt.p3.sg}] [was{\sc.lp.nt}] [need{\sc.adv}] 
$\rightarrow$
[needed{\sc.lp.nt}] [is{\sc.clt.p3.sg}] [do{\sc.inf}]

(It needed to be done.)
}

\end{itemize}


% Tables:

\begin{table}

\begin{center}
\begin{tabular}{|l|rrr|}
\hline
\textbf{Dictionary} & \textbf{Paradigms} & \textbf{Entries} & \textbf{Forms} \\
\hline
Serbo-Croatian &  1,033 & 13,206 & 233,878 \\
Slovenian &  1,909 & 13,383 & 147,580 \\
\hline
Bilingual &  69 &  16,434 & -- \\
\hline
\end{tabular}
\caption{Statistics on number of lexicon entries for each of the dictionaries in the 
   system.}
\label{table:lexicons}
\end{center}

\end{table}

\begin{table}
\begin{center}
\begin{tabular}{|l|rr|}
\hline
 \textbf{Type}      & \texttt{hbs}$\rightarrow$\texttt{slv} & \texttt{slv}$\rightarrow$\texttt{hbs}\\
\hline
Disambiguation      &     194              &     28 \\
Lexical selection   &     --            &  42 \\
Transfer            &                47 &  98 \\
\hline

\end{tabular}
 \caption{Statistics on the number of rules in each direction. For the lexical selection rules, 
   the number indicates that there are 42 rules for each of the three standard varieties currently
   supported.}
\label{table:rules}
\end{center}
\end{table}

% EVALUATION 
\section{Evaluation}

\subsection{Lexical coverage}

\subsection{Quantitative}

\subsection{Qualitative}



% CONCLUSIONS AND FUTURE WORK
\section{Future work}

mention different word order, long phrases, difficult to write rules for

improve coverage, disambiguation (esp. for slv) and transfer rules

work on more Slavic language pairs (e.g. hbs-rus)

backport improvements in the HBS components to hbs-mak

keep up-to-date with latest politico-linguistic developments. Add in
 Montenegrin when the standard is agreed on.

small timeframe, disambiguation and transfer rudimentary

\section{Conclusions}


\section*{Acknowledgements}

Removed for review.
%The development of this language pair was funded as a part of the
%Google Summer of Code.\footnote{\url{http://code.google.com/soc/}}
%Many thanks to the language pair co-author Ale\v{s} Horvat and his
%mentor Jernej Vičič, and other Apertium contributors for their
%invaluable help and support.


%\nocite{*}

\bibliographystyle{lrec2006}
\bibliography{apertium-hbs-slv}

\end{document}

