\section{Design}
\subsection{The Apertium platform}
\nocite{forcada2011apertium}
The Apertium\footnote{\url{http://wiki.apertium.org/}} platform is a
modular machine translation system. The typical core layout consists
of a letter transducer morphological lexicon.\footnote{A list of
ordered pairs of word surface forms and their lemmatised
analyses.} The transducer produces cohorts\footnote{A cohort 
consists of a surface form and one or more readings containing the lemma of the 
word and the morphological analysis.} which are then subjected to a
morphological disambiguation process.
%
Disambiguated readings are then looked up in the bilingual dictionary,
which gives the possible translations for each reading. These
are then passed through a lexical-selection module \cite{tyers12a}, 
which applies rules that select the most appropriate translation
for a given source-language context.
After lexical selection, the readings, which are now pairs of source
and target language lexical forms are passed through a 
syntactic transfer module that performs word reordering, deletions,
insertions, and basic syntactic chunking.
%
The final module is another letter transducer which generates
surface forms in the target language from the bilingual transfer
output cohorts.

\subsection{Constraint Grammar}
This language pair uses a Constraint Grammar (CG)
module\footnote{Implemented in the CG3 formalism, using the
  \texttt{vislcg3} compiler, available under GNU GPL. For a detailed
  reference see: \url{http://beta.visl.sdu.dk/cg3.html}} for
disambiguation \cite{karlsson1995constraint}. The CG formalism consists of hand-written rules that
are applied to a stream of tokens. Depending on the morphosyntactic
context of a given token the rules select or exclude readings of a
given surface form, or assign additional tags.
