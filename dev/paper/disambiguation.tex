
\subsection{Disambiguation}
Though for both languages there exists a number of tools for
morphological taggin and disambiguation (\todo{reference some}), there
are none freely available. Likewise, the adequatly tagged corpora are
mostly non-free (\todo{reference some non-free corpora, like
  1984.,...}). \todo{See on the taggers/corpora for Slovene}\todo{See
  on the taggers/corpora for BCMS}.  Since both BCMS Slovene are
highly inflected languages, the automatically trained statistical
tagger canonically used in Apertium language pairs would not give
satisfactory results. For this reason we chose to use solely
Constraint Gramar (CG) for disambiguation. The CG module does not
provide complete disambiguation, so in the case of any remaining
ambiguity the system picks the first output analysis.

Due to the similarities between the languages, we were able to
reuse much of the rules developed earlier for BCMS. Following are
a few disambiguation rules examples:

\begin{itemize}

\item Simple adverb vs. adjective rule:

\sentenceexample{
Ja često jedem ribu. $\leftrightarrow$ Jaz pogosto jedem ribo.

(I eat fish often.)
}

For this phrase the morphological analyser gives:

{\small
\begin{Verbatim}
"<Ja>"
    "free" prn pers p1 mfn sg nom 
"<često>"
    "često" adv
;   "čest" adj pst nt sg nom ind
;   "čest" adj pst nt sg nom def
;   "čest" adj pst nt sg voc ind
;   "čest" adj pst nt sg voc def
;   "čest" adj pst nt sg acc ind
;   "čest" adj pst nt sg acc def
"<jedem>"
    "jesti" vb imperf tv pres p1 sg
;   "jesti" vb imperf ref pres p1 sg
;   "jesti" vb imperf iv pres p1 sg
"<ribu>"
     "riba" n f sg acc 
\end{Verbatim}
}

The rule used to disambiguate the adverb \emph{često} in this phrase

{\small\begin{verbatim}
    SELECT Adv IF 
        (0 Adv OR A) 
        (1C V)
\end{verbatim}
}

selects the adverb reading in an adverb/adjective ambiguity if the
word following is unambiguously a verb.

\item Preposition based case disambiguation:

\sentenceexample{
Za našo ljubo staro mater. $\leftrightarrow$ Za našu dragu staru majku.

(For our dear old mother.)
}

Noun phrases in both languages typically generate a great number of
ambiguities.

{\small
\begin{Verbatim}
    "<Za>"
         "za" pr acc 
    ;    "za" pr gen
    ;    "za" pr ins
    "<našo>"
         "naš" prn pos p1 f sg acc 
    ;    "naš" prn pos p1 f sg ins
    "<ljubo>"
         "ljub" adj f sg acc ind 
    ;    "ljubo" adv sint
    ;    "ljub" adj nt sg nom ind
    ;    "ljub" adj nt sg acc ind
    ;    "ljub" adj f sg ins ind
    "<staro>"
         "star" adj f sg acc ind 
    ;    "staro" adv sint
    ;    "star" adj f sg ins ind
    ;    "star" adj nt sg nom ind
    ;    "star" adj nt sg acc ind
    "<mater>"
         "mati" n f sg acc 
    ;    "mati" n f du gen
    ;    "mati" n f pl gen
\end{Verbatim} 
}

First the rule

{\small
\begin{Verbatim}
    REMOVE Pr + $$Case IF 
        (1 Nominal - $$Case)
\end{Verbatim}
}

removes a case reading from a preposition if it is not followed by an
adjective, noun or a pronoun in the same case, and then the rule

{\small
\begin{Verbatim}
    REMOVE Nominal + $$Case IF
        (NOT -1 Prep + $$Case) 
        (NOT -1 Modifier + $$Case)
\end{Verbatim}
}

in several passes removes the incorrect cases from the nouns and
modifiers in the phrase.

\item Noun heuristics:

\sentenceexample{
...izvršavanje dužnosti predstavnice...

(...doing the duty of a representative...)
}

Frequently in both languages if there is a sequence of nouns, the
second noun is in genitive.

{\small
\begin{Verbatim}
    "<izvršavanje>"
        "izvršavanje" n nt sg nom 
    ;   "izvršavanje" n nt sg voc
    ;   "izvršavanje" n nt sg acc
    "<dužnosti>"
        "dužnost" n f sg gen
    ;   "dužnost" n f sg voc
    ;   "dužnost" n f pl voc
    ;   "dužnost" n f sg loc
    ;   "dužnost" n f sg dat
    ;   "dužnost" n f pl acc
    ;   "dužnost" n f sg ins
    ;   "dužnost" n f pl nom
    ;   "dužnost" n f pl gen
    "<predstavnice>"
        "predstavnica" n f sg gen
    ;   "predstavnica" n f sg voc
    ;   "predstavnica" n f pl voc
    ;   "predstavnica" n f pl nom
    ;   "predstavnica" n f pl acc
\end{Verbatim}
}

This simple heuristic selects a genitive reading of a noun after a noun:

{\small
\begin{Verbatim}
    SELECT: N + Gen IF (-1 N)
\end{Verbatim}
}
\end{itemize}

