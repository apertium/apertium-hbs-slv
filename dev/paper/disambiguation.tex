\subsection{Disambiguation}
Though for both languages there exists a number of tools for
morphological taggin and disambiguation (\todo{reference some}), there
are none freely available. Likewise, the adequatly tagged corpora are
mostly non-free (\todo{reference some non-free corpora, like
  1984.,...}). \todo{See on the taggers/corpora for Slovene}\todo{See
  on the taggers/corpora for BCMS}.  Since both BCMS Slovene are
highly inflected languages, the automatically trained statistical
tagger canonically used in Apertium language pairs would not give
satisfactory results. For this reason we chose to use solely
Constraint Gramar (CG) for disambiguation. The CG module does not
provide complete disambiguation, so in the case of any remaining
ambiguity the system picks the first output analysis.

Due to the similarities between the languages, we were able to
reuse much of the rules developed earlier for BCMS. Following are
examples of disambiguation rules:

\todo{Disambiguation examples}

\begin{itemize}
\item Preposition based case disambiguation

\sentenceexample{
    Za mojo mater. $\leftrightarrow$ Za moju majku.

    [Diseases{\sc.acc}] [easily{\sc.adv}] [cause{\sc.p3.sg}] [viruses{\sc.nom}] $\rightarrow$ [Diseases{\sc.acc}] [can{\sc.p3.sg}] [cause{\sc.inf}] [viruses{\sc.nom}]

    (Viruses can cause diseases.)
  }

\end{itemize}

