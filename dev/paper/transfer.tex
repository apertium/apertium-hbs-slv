\subsection{Transfer}

The BCMS and Slovene languages are very closely related, and their
morphologies are extremely similar. Most of non-technical transfer
rules are thus written only for rare syntactic differences. These are
mostly about clitic ordering, and different noun case usage.

Following are examples of transfer rules, which also illustrate some
contrastive characteristics of the languages:

\note{Pravila funkcioniraju u oba smjera pa objasniti oba smjera}

\begin{itemize}
\item The future tense:
\enumsentence{
Ja ću gledati\footnote{The encliticised future tense forms (gledat ću / gledaću) are handled equally.} $\rightarrow$ Jaz bom gledal

[I] [will{\sc.clt.p1.sg}] [watch{\sc.inf}] $\rightarrow$ [I] [will{\sc.clt.p1.sg}] [watch{\sc.pres.lp.m.sg}]

(I will watch.)
}
\end{itemize}

\todo{je treba $\rightarrow$ treba}

\todo{ni bilo treba $\rightarrow$ nije trebalo}

\todo{još jedan s treba}

\todo{clitic ordering}

\todo{lahko $\rightarrow$ može primjer 1}

\todo{lahko $\rightarrow$ može primjer 2}
