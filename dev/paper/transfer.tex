%% Redefiniramo enum sentence da bude manji, jer primjeri ispadaju preveliki
\newcommand{\sentenceexample}[1]{{\small\enumsentence{#1}}}

\subsection{Transfer}

The BCMS and Slovene languages are very closely related, and their
morphologies are extremely similar. Most of non-technical transfer
rules are thus written only for rare syntactic differences. These are
mostly about clitic ordering, and different noun case usage.

Following are examples of transfer rules, which also illustrate some
contrastive characteristics of the languages:

\note{Pravila funkcioniraju u oba smjera pa objasniti oba smjera}

\begin{itemize}
% Futur
\item The future tense:
  \sentenceexample{
Gledal bom
$\leftrightarrow$ 
Gledat ću\footnote{The BCMS analyser covers the encliticised future tense forms (gledat ću / gledaću) as well.}

[watch{\sc.lp.m.sg}][be{\sc.clt.p1}] 
$\leftrightarrow$ 
[watch{\sc.inf}][will{\sc.clt.p1}]

(I will watch.)
}

Both languages form the future tense in an analytic manner. While Slovene
uses a perfective form of the verb \emph{to be} combined with the l-participle (analogous
to BCMS future II), BCMS uses a cliticised form of the verb \emph{to
  want} combined with the infinitive. Unlike the infinitive, the
l-participle carries the information on the gender and number. Since
in this simplest form we have no way of inferring the gender of the
subject in the direction BCMS $\rightarrow$ Slovene the translation defaults
to masculine.

\item \emph{lahko} and \emph{moći}:
  \sentenceexample{
    Bolezni lahko povzročijo virusi $\leftrightarrow$ Bolesti mogu prouzročiti virusi

    [Diseases{\sc.acc}] [easily{\sc.adv}] [cause{\sc.p3.sg}] [viruses{\sc.nom}] $\rightarrow$ [Diseases{\sc.acc}] [can{\sc.p3.sg}] [cause{\sc.inf}] [viruses{\sc.nom}]

    (Viruses can cause diseases.)
  }

Unlike it's BCMS cognate \emph{lako}\footnote{\emph{easily}} the adverb \emph{lahko} in
Slovene, when combined with a verb has an additional meaning of \emph{can be
  done}, expressed in BCMS with the modal verb \emph{moći}. Rules that
cover these type of phrases normalise the target verb to infinitive,
and transfer grammatical markers for number and person to the verb \emph{moći}.

\item \emph{lahko} and conditional:
  \sentenceexample{
Lahko bi napravili $\leftrightarrow$ Mogli bi napraviti

[easily{\sc.adv}] [would{\sc.clt.cnd}] [do{\sc.lp.pl}] $\rightarrow$ [Can{\sc.lp.pl}] [would{\sc.clt.cnd.p3.sg}] [do{\sc.inf}]

(We/they could do)
}

Another case of unevenness of morphological markers is the conditional
mood. The conditional marker in BCMS is the aorist form of the verb
\emph{to be}, and carries the information on person and
number. Slovene, and the majority of colloquial BCMS varieties, use
a frozen clitic form of the same verb\footnote{\emph{bi}, regardless of person and
number}. Thus in cases like this example, when it's impossible to
exactly infer the person and number the system defaults to the
colloquial form.

\item Lahko and conditional more complicated:
  \sentenceexample{
Mi bi lahko napravili $\leftrightarrow$ Mi bismo mogli napraviti

[We.{\sc.p1.pl}] [would{\sc.clt.cnd}] [easily{\sc.adv}] [do{\sc.lp.pl}]
$\rightarrow$ 
[We.{\sc.p1.pl}] [would{\sc.clt.cnd.p3.pl}] [can{\sc.lp.pl}] [do{\sc.inf}]

(We could do)
}

In this example the information on person and number is available
on the pronoun \emph{mi}, and can be copied in translation to the
conditional verb.

\item ko je bilo treba $\leftrightarrow$ kad je trebalo
  \sentenceexample{
    Ko je bilo treba $\leftrightarrow$ Kad je trebalo

    [Diseases{\sc.acc}] [easily{\sc.adv}] [cause{\sc.p3.sg}] [viruses{\sc.nom}] $\rightarrow$ [Diseases{\sc.acc}] [can{\sc.p3.sg}] [cause{\sc.inf}] [viruses{\sc.nom}]

    (Diseases can be caused by viruses.)
  }

\item mu je treba napraviti $\leftrightarrow$ treba mu napraviti
  \sentenceexample{
    Mu je treba napraviti $\leftrightarrow$ Treba mu napraviti

    [Diseases{\sc.acc}] [easily{\sc.adv}] [cause{\sc.p3.sg}] [viruses{\sc.nom}] $\rightarrow$ [Diseases{\sc.acc}] [can{\sc.p3.sg}] [cause{\sc.inf}] [viruses{\sc.nom}]

    (Diseases can be caused by viruses.)
  }

\end{itemize}
